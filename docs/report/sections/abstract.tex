\chapter{Sommario}
La Teoria del Controllo è una disciplina in rapido sviluppo, vista
la sua applicabilità in molti ambiti.
Il punto centrale di questa tesi è lo studio del \emph{pendolo su rotaia} e,
in particolare, di come sia possibile realizzare una strategia di controllo capace
di mantenere il pendolo invertito in posizione verticale.
La strategia deve essere efficace sia per il modello teorico del sistema, sia per
la sua costruzione pratica in laboratorio.
La strategia finale che ho sviluppato usa il metodo di controllo non lineare di
Ljapunov per alzare il pendolo e poi passa al regolatore lineare quadratico,
applicato al sistema linearizzato attorno alla posizione verticale,
per stabilizzarlo.
La strategia si è rivelata efficace per il sistema reale e il comportamento osservato
è compatibile con quanto previsto dal modello.

\vspace{10cm}

\section*{Ringraziamenti}

Ringrazio la mia famiglia.\\
Ringrazio i colleghi che mi hanno aiutato in questo percorso.


\vskip\onelineskip
\begin{flushleft}
    \sffamily
    \textbf{Matteo Bonacini}
    \\
    Bologna,\MONTH\the\year
\end{flushleft}