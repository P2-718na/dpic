\section{Obiettivi}
Il pendolo invertito su rotaia è un sistema che,
sebbene sia facile da modellare, presenta alcune
caratteristiche che rendono interessante studiarne la
controllabilità.
Il sistema ha due punti di equilibrio; è
\emph{non lineare} e \emph{sottoattuato} (ovvero, un solo controllo
deve gestire due gradi di libertà).
Lo studio che sviluppo in questo capitolo ha il duplice obiettivo di:
\begin{enumerate}
    \item trovare una strategia per stabilizzare il pendolo attorno al punto di equilibrio instabile, in modo che sia resistente alle perturbazioni (\emph{strategia di stabilizzazione}),
    \item trovare una strategia per portare il pendolo in prossimità del punto di equilibrio instabile, partendo dalla configurazione stabile (\emph{strategia di swing-up}).
\end{enumerate}
Applicherò poi questo studio ad un sistema reale che ho costruito, 
in modo da confrontare i risultati teorici con quelli pratici.