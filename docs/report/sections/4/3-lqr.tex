\section{Strategia di stabilizzazione}
\label{sec:strategia-stabilizzazione}
Per stabilizzare il sistema uso il regolatore lineare quadratico
descritto nel paragrafo~\ref{sec:controllo-ottimale},
applicato alle equazioni del moto~\eqref{eq:moto-sistema}
linearizzate attorno al punto di equilibrio $\theta = 0$.

\subsection{Linearizzazione delle equazioni}
Risolvo le equazioni~\eqref{eq:moto-sistema} per $\ddot q$ e $\ddot \theta$:
\begin{subequations}
    \begin{empheq}[left=\empheqlbrace,right=_.]{align}
        \ddot q &= \frac{
            lm\sin\theta \cdot \left(I\dot \theta^2-glm\cos\theta\right)+If
        } {
            I(m+M) - (lm\cos\theta)^2
        } \label{eq:moto-carrello-q}\\
        \ddot \theta &= \frac{
            lm\left[\sin\theta \cdot (lm\dot\theta^2 \cos\theta - gm - gM) + f\cos\theta  \right]
        }{
            (lm\cos\theta)^2 - I(m+M)
        } \label{eq:moto-carrello-theta}
    \end{empheq}
    \label{eq:moto-carrello-totale}
\end{subequations}
\todo{Ricontrolla equazioni e nomi parametri}
Il sistema~\eqref{eq:moto-carrello-totale} ha due gradi di libertà,
quindi lo spazio delle fasi è $4$-dimensionale.
Applico la sostituzione
\begin{equation*}
    \left\{
    \begin{aligned}
        \dot q &\mapsto v_q \\
        \ddot q &\mapsto \dot v_q \\
        \dot \theta &\mapsto v_\theta \\
        \ddot \theta &\mapsto \dot v_\theta
    \end{aligned}
    \right.
    _.
\end{equation*}
Gli elementi dello spazio delle fasi sono i vettori colonna
\begin{equation*}
    \b x = (q, v_q, \theta, v_\theta)^{\T}
\end{equation*}
e l'equazione del moto è nella forma~\eqref{eq:sistema-non-lineare}:
\begin{equation*}
    \dot {\b x}(t) = \left(
    \begin{aligned}
        &v_q \\
        &\dot v_q = \text{ equazione~\eqref{eq:moto-carrello-q}}\\
        &v_\theta \\
        &\dot v_\theta  = \text{ equazione~\eqref{eq:moto-carrello-theta}}
    \end{aligned}
    \right) = F(\b x, f).
\end{equation*}

Applico la~\eqref{eq:formula-linearizzazione} e ottengo la forma linearizzata
del sistema
\begin{equation*}
        \dot {\b x} = A\b x + Bf \\
\end{equation*}
con
\begin{equation*}
    A = \left(
            \begin{array}{cccc}
                0&1&0&0\\
                0&0&\frac{g(lm)^2}{d}&0\\
                0&0&0&1\\
                0&0&\frac{glm(m+M)}{-d}&0\\
            \end{array}
        \right),\
    B = \left(\begin{array}{c}0\\\frac{I}{-d}\\0\\\frac{lm}{d}\end{array}\right)f,
\end{equation*}
dove $d = (lm)^2-I(m+M)$.
\todo{ricontrolla numeri.}

Concludo osservando che il sistema è controllabile.
Infatti, applicando la \autoref{def:matrice-controllabilità}
ottengo
\begin{equation*}
    \mathcal C = \left(
        \begin{array}{cccc}
            0&\frac I {-d}&0&\frac{lm} d\\
            \frac I {-d}&0&\frac{lm} d&0\\
            0&\frac{g(lm)^3}{d^2}&0&\frac{g(lm)^2(m+M)}{-d^2}\\
            \frac{g(lm)^3}{d^2}&0&\frac{g(lm)^2(m+M)}{-d^2}&0\\
        \end{array}
        \right)
\end{equation*}
che ha rango massimo quando $l, m \neq 0$.

%todo here
\subsection{Sviluppo del controllo ottimale}
Fisso i coefficienti della funzione costo.
In $\b x = 0$ la matrice Hessiana dell'energia cinetica è
semidefinita positiva, mentre l'Hessiana del potenziale è
semidefinita negativa.
Quindi l'Hessiana di $T - V$ è semidefinita positiva e
la posso usare come matrice di costo $Q$,
così da dare alla quantità
da minimizzare il significato di energia mediata sul tempo.
Tuttavia, visto che le equazioni del moto non dipendono da $q$,
sono costretto ad introdurre un potenziale fittizio
$q_{11} > 0$ per far sì che il sistema tenda allo
stato dove $q = 0$.
$R$ va scelto in base alla forza massima che può sopportare
il motore; per ora fisso $R = (r_{11})$ con $r_{11} > 0$.
\todo{O qui o nella sezione risultati, ci sta di mostrare come varia la soluzione in funzione di R. Vedere se influisce solo sul gain del motore...}

Svolgendo il calcolo trovo
\begin{equation}
    Q = \left(
    \begin{array}{cccc}
        q_{11} & 0 & 0 & 0 \\
        0 & m+M & 0 & lm \\
        0 & 0 & glm & 0 \\
        0 & lm & 0 & I
    \end{array}
    \right), \
    R = \left(
    r_{11}
    \right).
    \label{eq:q-r-matrices}
\end{equation}
%todo molto importante controlla che le equazioni del moto sul notebook
%siano corrette

La strategia di controllo è data da
\begin{equation*}
    f = K\b x
\end{equation*}
dove $K$ è dato dalla~\eqref{eq:riccati-K}.