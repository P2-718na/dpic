\section{Modello del sistema}
Nel paragrafo precedente ho detto che l'unica interazione possibile tra il sistema
e l'esterno è l'applicazione di una forza sul carrello.
Nella pratica, questo significa che è presente un motore vincolato al carrello
in qualche modo, come mostrato in figura~\ref{fig:pic-real}.
Mentre la natura del vincolo non è interessante, dobbiamo invece prestare
particolare attenzione al tipo di motore utilizzato.
Un motore esercita una forza che dipende sia da un segnale di controllo esterno,
sia dallo stato interno dello stesso.
È quindi conveniente separare lo studio del sistema in due:
pendolo-carrello e motore.

\subsection{Pendolo e carrello}
È immediato ricavare le equazioni del moto usando l'approccio Lagrangiano.
Inizio fissando alcuni parametri che possono essere calcolati o misurati
sperimentalmente; sono riportati in tabella~\ref{tab:parametri}.
Posso quindi scrivere l'espressione per l'energia cinetica $T$
e potenziale $V$ del sistema:

\begin{equation*}
    \begin{aligned}
        T &= \frac 1 2 M  \dot q^2 + \frac 1 2 m v_{cm}^2 + \frac 1 2 (J - lm^2)\dot \theta^2 \\ %todo
        V &= mgL \cos\theta
    \end{aligned}
    \hspace{20pt} \text{con } q \in \R, \theta \in ]-\pi, +\pi]
    \label{eq:energy}
\end{equation*}
dove $v_{cm}$ è la velocità del centro di massa del pendolo, data da:
\begin{equation*}
    v^2_{cm} = (\dot x + l \dot \theta \cos \theta)^2 + (+ l \dot \theta \sin \theta)^2.
\end{equation*}
La Lagrangiana $\mathcal L$ è:
\begin{equation*}
    \mathcal L = T - V
\end{equation*}
e posso impostare le equazioni di eulero
\begin{equation*}
    \begin{aligned}
        \frac{\partial^2}{\partial t \partial \dot q} \mathcal L - \partiald q \mathcal L &= f \\
        \frac{\partial^2}{\partial t \partial \dot \theta} \mathcal L  - \partiald \theta \mathcal L &= 0
    \end{aligned}
    \label{eq:eulero}
\end{equation*}
da cui ricavo le equazioni del moto del sistema:
\begin{equation}
    \begin{aligned}
        \ddot x &= \frac {-lm \ddot \theta \cos \theta + lm \dot \theta^2 \sin \theta + f} {m + M} \\
        \ddot \theta &= \frac {lm (g\sin \theta - \cos \theta }{J} \ddot x.
    \end{aligned}
    \label{eq:moto-sistema}
\end{equation}
In questo modello ho trascurato gli attriti tra il carrello e la rotaia e tra
il pendolo e il perno.

Per concludere devo studiare i punti di equilibrio del sistema.
L'unica variabile che compare nel potenziale è $\theta$, quindi il calcolo è
immediato.
\begin{equation*}
    \left. \frac \partial {\partial \theta}\right |_{V=V_{eq}} V =  0 \implies V_{eq} = \{0, \pi\}.
\end{equation*}
È immediato anche studiarne al stabilità:
\begin{align*}
    \theta = 0 \mapsto \text{instabile} \\
    \theta = \pi \mapsto \text{stabile}.
\end{align*}

\bgroup
\renewcommand{\tabularxcolumn}[1]{>{\arraybackslash}m{#1}}
\renewcommand\arraystretch{1.5}
\begin{table}[h]
    \centering
    \begin{tabularx}{\textwidth}{| gc | X |}
        \noalign{\hrule height 2pt}

        \rowcolor{Black}%

        \multicolumn{1}{=c}{\rowstyle{\bfseries\sffamily \color{White}} Parametro/Variabile} & \multicolumn{1}{+c}{ Descrizione} \\
        \hline
        $g$ & Accelerazione di gravità. \\
        \hline
        $M$ & Massa del carrello. \\
        \hline
        $m$ & Massa del pendolo. \\
        \hline
        $L$ & Distanza tra il centro di massa del pendolo e il punto di rotazione. \\
        \hline
        $I$ & Momento d'inerzia del pendolo calcolato rispetto al punto di rotazione. \\
        \hline
        $q$ & Posizione del carrello rispetto all'origine. \\
        \hline
        $\theta$ & Angolo del pendolo rispetto alla verticale. \\
        \hline
        $f$ & Forza agente sul carrello. \\
        \noalign{\hrule height 2pt}
    \end{tabularx}
    \caption{Descrizione di parametri e variabili del sistema carrello-pendolo.}
    \label{tab:parametri}
\end{table}
\egroup

\subsection{Motore}
\todo{fix reference}
\ref{https://homepages.laas.fr/lzaccari/seminars/DCmotors.pdf}
Per questa applicazione è sufficiente usare un motore DC a spazzole. Il motore è attivato applicando una differenza di potenziale $U$ tra le due armature e il verso di rotazione dipende dal segno della differenza di potenziale. Variando $U$ si ottiene grossolanamente un controllo sulla velocità di rotazione. Io voglio controllare la forza che il motore esercita sul carrello e devo quindi trovare la relazione tra differenza di potenziale e coppia $\tau$.
Fisso alcuni parametri relativi al motore, riportati in tabella \ref{tab:parametri-motore}.

\bgroup
\renewcommand{\tabularxcolumn}[1]{>{\arraybackslash}m{#1}}
\renewcommand\arraystretch{1.5}
\begin{table}[t]
    \centering
    \begin{tabularx}{\textwidth}{| gc | X |}
        \noalign{\hrule height 2pt}

        \rowcolor{Black}%
        \multicolumn{1}{=c}{\rowstyle{\bfseries\sffamily \color{White}} Parametro/Variabile} & \multicolumn{1}{+c}{ Descrizione} \\
        \hline
        $u$ & \ddp tra le armature. \\
        \hline
        $J$ & Corrente che scorre nelle armature. \\
        \hline
        $\tau$ & Coppia esercitata dal motore. \\
        \hline
        $f$ & Forza agente sul carrello. \\
        \hline
        $\dot q$ & Velocità del carrello. \\
        \noalign{\hrule height 2pt}
    \end{tabularx}
    \caption{Descrizione di parametri e variabili del motore.}
    \label{tab:parametri-motore} %todo qui manca la descrizion2e di un tot di parametri.i
\end{table}
\egroup

Un motore DC è regolato dalle equazioni: \todo{qui devo ritrovare il pdf che spiega tutto bene e decidere se fare io i disegni o se citare semplicemente altri paper}
\begin{equation*}
    \left\{
    \begin{aligned}
        u &= L_a \dot J + R_a J + K_e \omega \\
        \tau &= K_m J
    \end{aligned}
    \right.
    .
\end{equation*}

La coppia è proporzionale alla corrente che scorre nel motore. Si possono realizzare diversi circuiti di alimentazione che controllano direttamente la corrente ma il sistema che ho realizzato nella sezione \ref{sec:sistema-reale} può solo controllare il voltaggio. In più, il sistema non ha un sensore per misurare la corrente che passa nel motore.
Per trovare una soluzione al mio problema inizio risolvendo il sistema per $U$:
\begin{equation*}
    \begin{aligned}
        u &= L_a \partiald t \left(\frac \tau {K_m}\right) + R_a \frac \tau {K_m} + K_e \omega \\
        &= \frac {L_a} {K_m} \dot\tau + \frac{R_a}{K_m}  \tau + K_e \omega.
    \end{aligned}
\end{equation*}
In linea di principio questo è tutto quello che mi serve per controllare il motore, tuttavia, il termine che contiene $\dot \tau$ ha un coefficiente che non è facile da misurare con gli strumenti che ho a disposizione. Scelgo quindi di trascurare questo termine. Questa scelta è giustificata dal fatto che il segnale di controllo $\tau$ cambia solamente ogni $\Delta t$ secondi e posso scegliere $\Delta t$ in modo che sia maggiore del tempo del transiente dovuto a $\Delta \tau$.

Ricordando che $\tau$ e $\omega$ sono proporzionali a $f$ e $\dot q$, trovo che questo modello dipende solo da due parametri determinabili sperimentalmente, $A$ e $B$:
\begin{equation}
    u = A f + B \dot q.
    \label{eq:caratteristica-motore}
\end{equation}
La determinazione di questi parametri è descritta nel paragrafo \ref{subsec:parametri-motore}.