\section{Strategia di stabilizzazione}
\label{sec:strategia-stabilizzazione}
Per stabilizzare il sistema uso il Linear Quadratic Regulator, così come ho descritto nel capitolo \ref{sec:linear-control}. Ho già ricavato le equazioni del moto \eqref{eq:moto-sistema} e per poter usare LQR devo solamente linearizzarle attorno al punto $\theta = 0$. La risoluzione dell'equazione di Riccati (??) \todo{qui ci dovrò linkare equazione che avrò scritto nei capitoli sopra} è fatta numericamente usando Mathematica.

Devo però tenere a mente due considerazioni:
\begin{itemize}
    \item Non ho nulla che mi garantisca che il sistema linearizzato si comporti come il sistema non lineare. Dovrò quindi verificare a posteriori se il risultato che otterrò funzionerà o meno. \todo{in realtà, lo strogatz dice che se ci sono termini al primo ordine, allora ho la garanzia che il sistema si comporti così vicino al punto di equilibrio. Se ci fossero solo termini al secondo ordine, non potrei linearizzare il sistema...}

    \item Le equazioni che ho trovato descrivono un sistema a tempo continuo. Nel sistema reale, sia i sensori che il motore hanno un certo tempo di risposta $\Delta t$. Dovrò quindi trasformare il mio modello da tempo continuo a tempo discreto.
\end{itemize}

\subsection{Linearizzazione delle equazioni}
Considero le equazioni \eqref{eq:moto-sistema}. Scelgo di linearizzarle usando il metodo della Jacobiana descritto in [Strogatz]. Le risolvo per $\ddot \theta$ e $\ddot q$:
\begin{equation*}
    \left\{
    \begin{aligned}
        \ddot q &= &\frac{2m\sin(\theta(t))\left(3g\cos(\theta(t))-2l\theta'(t)^2\right)-8u(t)} {3m\cos(2\theta(t))-5m-8M} \\
        %
        \ddot \theta &= &\frac{3\sin(\theta(t))\left(lm\theta'(t)^2\cos(\theta(t))-2g(m+M)\right)+6u(t)\cos(\theta(t))}{l\left(3m\cos^2(\theta(t))-4(m+M)\right)}
    \end{aligned}
    \right.
\end{equation*}
\todo{Ricontrolla equazioni e nomi parametri}
Ora riduco l'ordine del sistema applicando la sostituzione:
\begin{align}
    \dot q &\mapsto v_q \\
    \ddot q &\mapsto \dot v_q \\
    \dot \theta &\mapsto v_\theta \\
    \ddot \theta &\mapsto \dot v_\theta.
\end{align}

L'equazione del moto $\dot {\b x}(t)$ del sistema è quindi una funzione di cinque variabili:
\begin{equation*}
    \dot {\b x}(t) = \left(
    \begin{aligned}
        v_q \\
        \dot v_q \\
        v_\theta \\
        \dot v_\theta
    \end{aligned}
    \right) = F(v_q, \dot v_q, v_\theta, \dot v_\theta, f).
\end{equation*}
Procedo calcolando la matrice Jacobiana di F attorno al punto di equilibrio. Ricordo che ho scelto le variabili del mio sistema in modo che questo coincida con $\b x = 0$.
\begin{equation*}
    \begin{aligned}
        J_F(0)_{ij} &= \partialdd {x_j}{\b x = 0} F_i(\b x) \\
        &= \left(\begin{array}{ccccc}0&1&0&0&0\\0&0&-\frac{3gm}{m+4M}&0&\frac{4}{m+4M}\\0&0&0&1&0\\0&0&\frac{6g(m+M)}{l(m+4M)}&0&-\frac{6}{lm+4lM}\\\end{array}\right) .
        %
    \end{aligned}
\end{equation*}
\todo{anche qui controlla equazione}
L'equazione del moto è data da:
\begin{equation*}
    \dot {\b x} \approx J_F(0) \left( \begin{aligned}
                                          &q \\
                                          &v_q \\
                                          &\theta \\
                                          &v_\theta \\
                                          &f
    \end{aligned}  \right)
\end{equation*}
e posso usare le proprietà del prodotto righe per colonne per separare $J_F(0)$ in due matrici $A$ e $B$ e ricondurmi all'equazione ?? \todo{qui dovrò fare riferimento all'equazione che descrive il sistema}
\begin{equation*}
    \begin{aligned}
        \dot {\b x} &= A\b x + Bf \\
        &= \left(\begin{array}{cccc}0&1&0&0\\0&0&-\frac{3gm}{m+4M}&0\\0&0&0&1\\0&0&\frac{6g(m+M)}{l(m+4M)}&0\\\end{array}\right) \b x + \left(\begin{array}{c}0\\\frac{4}{m+4M}\\0\\\frac{-6}{lm+4lM}\end{array}\right)f.
    \end{aligned}
\end{equation*}
\todo{qui ci vanno calcoli sulla controllabilità del sistema}.

\subsection{Coefficienti per l'ottimizzazione}
Per risolvere il problema di controllo ottimale, devo fissare i coefficienti della funzione costo.
Questi coefficienti possono essere scelti arbitrariamente ed è naturale prendere $Q$ uguale alla matrice metrica del sistema, in modo da dare un significato fisico alla quantità che vogliamo minimizzare (l'integrale dell'energia sul tempo). $R$ va scelto in base a quanto "scattante" deve essere il motore. \todo{O qui o nella sezione risultati, ci sta di mostrare come varia la soluzione in funzione di R. Vedere se influisce solo sul gain del motore...}.
\begin{equation*}
    A = \left(
    \begin{array}{cccc}
        a_{11} & a_{12} & a_{13} & a_{14} \\
        a_{21} & a_{22} & a_{23} & a_{24} \\
        a_{31} & a_{32} & a_{33} & a_{34} \\
        a_{41} & a_{42} & a_{43} & a_{44}
    \end{array}
    \right).
\end{equation*}
\begin{equation*}
    R = \left(
    r_{11}
    \right).
\end{equation*}

%todo
\todo{calcolo}
Qui ci va il calcolo della matrice metrica e devo mostrare cosa sono i coefficienti di $A$.


\subsection{Gain del controller}
La strategia di controllo è data quindi da
\begin{equation*}
    f = -K\b x
\end{equation*}
dove $K$ è una matrice data da:
\begin{equation*}
    K = \left(
    \begin{array}{c}
        0 \\
        0 \\
        0 \\
        0
    \end{array}
    \right).
\end{equation*}
Qui non saprei se scrivere altro... In realtà K dovrebbe saltare fuori nei risultati, visto che dipende da R che è arbitrario... Magari posso calcolare un po' di K diversi in funzione di R e vedere cosa salta fuori direttamente in questa sezione.