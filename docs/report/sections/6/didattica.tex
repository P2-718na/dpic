\section{Questo esperimento come metodo didattico}
In questo ultimo, breve paragrafo voglio sottolineare che avere a disposizione
un sistema reale su cui svolgere esperimenti è un ottimo metodo didattico.
Aver costruito questo sistema mi ha permesso di
accompagnare il gruppo di ricerca in fisica dei Sistemi Complessi
del Dipartimento di Fisica e Astronomia dell'Università di Bologna
all'edizione di Settembre 2023 della \emph{Notte dei Ricercatori} di Rimini~\cite{notteRicercator}.
Nonostante alcuni problemi tecnici\footnotemark, siamo riusciti a coinvolgere con facilità un pubblico
vasto, in cui rientravano sia bambini delle elementari sia adulti di
varia formazione.
È infatti dimostrato~\cite{Hake_1998} che il \emph{coinvolgimento} professore-studenti
è un fattore chiave per l'efficacia dell'apprendimento.
Non ho scattato foto da inserire in questo testo,
ma sul canale YouTube dell'Università di Bologna~\cite{youtube} è stato caricato un breve
video che presenta la nostra attività.

\footnotetext{I due microcontrollori comunicano wireless usando la stessa lunghezza d'onda
delle reti Wi-Fi. Nel luogo della presentazione erano presenti troppe reti Wi-Fi
e abbiamo dovuto spostarci per far sì che l'esperimento funzionasse.
}