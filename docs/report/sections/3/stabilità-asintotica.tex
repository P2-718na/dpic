\section{Controllabilità asintotica}
Introduco la definizione di controllabilità asintotica, analoga alla
definizione~\ref{def:controllabilità} ma per il tempo tendente all'infinito.
\begin{definition}
    Dato un problema di controllo, siano
     $\b x_0, \b x_1 \in \mathcal V \subseteq \Sigma$ punti dello spazio delle fasi.
    Uso la notazione $\xi(t) = \phi^t(\b x_0, \nu)$.
    Se per ogni $t \in[0, +\infty[ \subseteq \mathcal T$
    esiste una funzione di controllo $\nu \in  \mathcal U^{[0, +\infty[ \subseteq \mathcal T}$ tale che
    \begin{align*}
       \lim_{s \to +\infty} &\xi(s) = \b x_1,\\
       &\xi(t) \in \mathcal V,
    \end{align*}
    allora si dice che $\b x_0$ può essere \textbf{controllato asintoticamente} verso
    $\b x_1$ \textbf{senza lasciare} $\mathcal V$.
    \label{def:x-controllabile-a-y}
\end{definition}
Se nella definizione~\ref{def:x-controllabile-a-y} $\mathcal V = \Sigma$,
si dice solamente che
$\b x_0$ \emph{può essere controllato asintoticamente} verso $\b x_1$.
\begin{definition}
    Dato un problema di controllo con $\b x_0 \in \Sigma$ si dice che
    \begin{itemize}
        \item Il problema è \textbf{localmente asintoticamente controllabile}
        se per ogni intorno $\mathcal V$ di $\b x_0$ esiste un intorno
        $\mathcal W \subseteq \mathcal V$ di $\b x_0$ tale che ogni $\b x \in \mathcal W$
        può essere controllato asintoticamente a $\b x_0$ senza lasciare $\mathcal V$.

        \item Il problema è \textbf{globalmente asintoticamente controllabile} se
        è localmente asintoticamente controllabile con $\mathcal W = \Sigma$.
    \end{itemize}
    \label{def:controllabilità-asintotica}
\end{definition}
Nella definizione~\ref{def:controllabilità-asintotica} il termine \emph{controllabile}
serve per sottolineare che si sta considerando un problema di controllo.
Per sistemi senza controlli, si dice che il sistema è \emph{asintoticamente stabile}, usando la
stessa definizione.

Per un sistema lineare, la proprietà di
essere controllabile asintoticamente coincide con l'avere tutti gli autovalori
con parte reale negativa, come descritto nella sezione~\ref{subsec:comportamento-asintotico}.
% todo cite sontag