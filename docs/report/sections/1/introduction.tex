
\title{Introduzione}
\maketitle
\label{sec:intro}


\section{Teoria del Controllo}
La Teoria del Controllo è una disciplina che ha origini antiche,
tracciabili fino all'era degli antichi greci,
e che si è sviluppata formalmente solo nel corso degli ultimi due secoli~\cite{history}.
Ancora oggi è in rapido sviluppo e presenta una serie di problemi aperti~\cite{blondel1999open}.

L'obiettivo alla base della Teoria è
dimostrare la possibilità di modificare l'evoluzione
temporale di un sistema fisico per ottenere un 
obiettivo specifico, collegato allo stato del sistema
considerato.
Un essere umano affronta problemi simili
costantemente e, spesso, gli appaiono banali.
Questa banalità apparente è frutto
della grande capacità di adattamento del
cervello che, automaticamente,
esercita un Controllo.

Basti prendere come esempio \emph{l'atto di andare in bicicletta}.
Un mito comune è che la bicicletta sia \emph{facile} da guidare perché
è stabilizzata dall'effetto giroscopico delle ruote, che la mantiene in verticale.
Questo è falso!
Uno studio~\cite{bicycle}
afferma che la bicicletta \emph{deve essere inerentemente instabile} per poter
essere usata da un essere umano; la facilità è data dal nostro cervello che,
in automatico, impara a correggere le instabilità.

I risultati della Teoria del Controllo si ritrovano in natura in processi di ogni tipo.
Dal metabolismo del singolo batterio, ai flussi di automobili nel traffico:
tutto questo necessita di Controllo.
In questa tesi mostrerò alcuni risultati della Teoria e
mostrerò il risultato della loro applicazione ad un sistema
meccanico che ho realizzato in laboratorio, il \emph{pendolo (invertito) su rotaia}.

\section{Organizzazione del testo}
Il testo è diviso in cinque capitoli, oltre al presente.
Nei capitoli~\ref{sec:linear-control} e~\ref{sec:nonlinear-control} introdurrò
i concetti teorici che ho usato nel mio studio e dimostrerò alcuni dei risultati
principali della Teoria, applicata ai sistemi lineari e non lineari.
Nel capitolo~\ref{sec:pic} enuncerò gli obiettivi del mio studio, svilupperò il modello
teorico del sistema e userò quanto ricavato nei capitoli
precedenti per trovare una strategia di controllo che raggiunga gli obiettivi.
Nel capitolo~\ref{sec:pic-irl} mostrerò come ho costruito il sistema in laboratorio
e nel capitolo~\ref{sec:results} elencherò i risultati che ho ottenuto applicando
il modello teorico al sistema reale.

La trattazione teorica in questo testo è largamente ispirata ai lavori di
Sontag~\cite{sontagMath} e Brunton~et.~al.~\cite{brunton_kutz_2019};
i concetti principali che presenterò in questo testo
sono un riadattamento di quanto scritto da loro.
Farò uso anche di altri risultati, che citerò di conseguenza.