
\title{Introduzione}
\maketitle
\label{sec:intro}


\section{Teoria del Controllo}
La Teoria del Controllo è una disciplina che ha origini antiche,
tracciabili fino all'era degli antichi greci,
%todo cite
e che si è sviluppata formalmente solo nel corso degli ultimi due secoli.
%todo cite, di nuovo

L'obiettivo alla base della Teoria è la necessità di poter alterare
lo stato di un sistema in modo arbitrario, entro i limiti fisici
del sistema stesso.
Anche se per un essere umano questo obiettivo può sembrare banale,
questa è solo un illusione data dall'estrema complessità del cervello umano:
spesso, attività che per noi appaiono semplici, contengono un
alto grado di complessità nascosta.
Basti prendere come esempio \emph{l'atto di andare in bicicletta}.
Un mito comune è che la bicicletta sia \emph{facile} da guidare perchè
è stabilizzata dall'effetto giroscopico delle ruote, che la mantiene in verticale.
Questo è falso!
Uno studio
\cite{bicycle}
%todo cite ALMOST EVERYONE can ride a bicycle, yet apparently no one knows how they do it.
afferma che la bicicletta \emph{deve essere inerentemente instabile} per poter
essere usata da un essere umano; la facilità è data dal nostro cervello che,
in automatico, impara a correggere le instabilità.

I risultati della Teoria si ritrovano in natura in processi di ogni origine.
Dal metabolismo del singolo batterio, ai flussi di automobili nel traffico:
tutto questo necessita di Controllo.
In questa tesi mostrerò alcuni risultati della Teoria e
mostrerò il risultato della loro applicazione ad un vero sistema
meccanico realizzato in laboratorio, il \emph{pendolo (invertito) su rotaia}.

\section{Organizzazione del testo}
Il testo è diviso in cinque capitoli, oltre al presente.
Nei capitoli~\ref{sec:linear-control} e~\ref{sec:nonlinear-control} introduco
i concetti teorici che servono per il mio studio e dimostro alcuni dei risultati
principali della Teoria, applicata ai sistemi lineari e non lineari.
Nel capitolo~\ref{sec:pic} enuncio gli obiettivi del mio studio, sviluppo il modello
teorico del sistema e uso quanto ricavato nei capitoli
precedenti per trovare una strategia di controllo che raggiunga gli obiettivi.
Nel capitolo~\ref{sec:pic-irl} mostro come ho costruito il sistema in laboratorio
e nel capitolo~\ref{sec:results} mostro i risultati che ho ottenuto applicando
il modello teorico al sistema reale.

La trattazione teorica in questo testo è largamente ispirata ai lavori di
Sontag e Brunton.
Nel costo del testo farò uso di altri risultati, che verranno citati di conseguenza.