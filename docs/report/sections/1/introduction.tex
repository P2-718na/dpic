
\title{Introduzione}
\maketitle
\label{sec:intro}


\section{Teoria del Controllo}
La Teoria del Controllo è una disciplina che ha origini antiche,
tracciabili fino all'era degli antichi greci
%todo cite
e che si è sviluppata formalmente solo nel corso degli ultimi due secoli.
%todo cite, di nuovo

L'obiettivo alla base della Teoria è la necessità di poter alterare
lo stato di un sistema in modo arbitrario, entro i limiti fisici
del sistema stesso.
Anche se per un essere umano questo obiettivo può sembrare banale,
questa è solo un illusione data dall'estrema complessità del cervello umano:
spesso, attività che per noi appaiono semplici, contengono un
alto grado di complessità nascosta.
Basti prendere come esempio \emph{l'atto di andare in bicicletta}.
Un mito comune è che la bicicletta è \emph{facile} da guidare perchè
l'effetto giroscopico delle ruote la stabilizza, mantenendola in verticale.
Questo è falso!
Uno studio
%todo cite ALMOST EVERYONE can ride a bicycle, yet apparently no one knows how they do it.
conferma che la bicicletta \emph{deve} essere inerentemente instabile per poter
essere usata da un essere umano; la facilità è data dal nostro cervello che,
in automatico, impara a correggere le instabilità.

I risultati della Teoria si ritrovano in natura in processi anche non meccanici.
Dal metabolismo del singollo batterio, ai flussi di automobili nel traffico:
tutto questo necessita di Controllo.
In questa tesi presento

\section{Organizzazione del testo}
Bla bla bla qui spiego un po' come ho organizzato il testo ma questo lo faccio alla fine
