\title{Controllo di sistemi lineari}
\maketitle
\label{sec:linear-control}

\paragraph{Introduzione}
La teoria del controllo applicata ai sistemi lineari è robusta e fornisce tutti gli strumenti necessari
per alterare lo stato di un sistema in modo arbitrario. In questo capitolo introdurrò alcuni
concetti di base per studiare il comportamento di un sistema lineare; mostrerò poi sotto quali
condizioni un sistema lineare è controllabile e infine proporrò un criterio per trovare una strategia
ottimale di controllo.


\iffalse
\section{Sistemi lineari}
La definizione di sistema lineare cambia leggermente a seconda di che si parli
di un sistema a tempo continuo o discreto.
È quindi opportuno enunciare la seguente definizione:

\todo{sistema numeraione delle definizioni}
\begin{definition}
    Un \textbf{insieme del tempo} $\mathcal T$ è un sottogruppo di $(\R, +)$.
\end{definition}
Nella pratica, $\mathcal T$ coinciderà sempre con $\R$ o con $\Z$, a seconda
che il sistema sia a tempo continuo o discreto, rispettivamente.
Inoltre, quando è definito un insieme del tempo $\mathcal T$, si assume che tutti gli
intervalli siano ristretti a $\mathcal T$.

\begin{definition}[Sistema]
    La quadrupla $\Sigma = (\mathcal T, \mathcal X, \mathcal U, \phi)$,
    dove:
    \begin{itemize}
        \item $\mathcal T$ è un insieme del tempo.
        %zzz
        \item $\mathcal X$ è un insieme non vuoto, detto \textbf{spazio delle fasi}.
        %
        \item $\mathcal U$ è un insieme non vuoto, detto \textbf{spazio dei controlli ammissibili}.
        %
        \item $\phi: D_\phi \to \mathcal X$ è un'applicazione, detta \textbf{flusso di fase},
        definita su: \\ \\
                $D_\phi :=
                   \left{
                       (\tau, \sigma, x, u) \mid
                       \sigma, \tau \in \mathcal T, \sigma < \tau, x \in \mathcal X, u \in \mathcal U^{[\sigma, \tau)}
                   \right}$, \\ \\
        \todo{non so come avere spacing per questa roba latex kys}
        dove $\mathcal U^{[\sigma, \tau)} := \left{u \mid u: [\sigma, \tau) \to \mathcal U \right}$.
    \end{itemize}
    È detta \textbf{sistema} se valgono le seguenti proprietà:
    \begin{itemize}
        \item \textbf{Non-banalità:}
        \item \textbf{Restrizione:}
        \item \textbf{Semigruppo:}
        \item \textbf{Identità:}
    \end{itemize}
\end{definition}
\fi
%qui sopra devo semplificare un attimo la definizione. Still, do definizione
%di sistema e di sistema lineare.
%la cosa migliore è partire da flusso di fase, dare definizione di campo vettoriale
%e vedere come passare da uuno all'altro.

\input{sections/2/sistemi-lineari}
\section{Controllabilità di un sistema lineare}

\subsection{Problema di controllo}
Introduco la definizione di \emph{problema di controllo}.
Questa si basa sulla definizione~\ref{def:sistema-dinamico} in cui,
assieme allo spazio delle fasi, viene aggiunto uno \emph{spazio dei controlli}.
Di seguito userò la notazione
\begin{equation*}
    \mathcal U^{\mathcal T},\ \text{con } \mathcal T \text{ intervallo}
\end{equation*}
per indicare lo spazio di tutte le funzioni da $\mathcal T$ a $\mathcal U$.

\begin{definition}
    La quadrupla $\left( \Sigma, \mathcal T, \mathcal U, \phi \right)$ in cui:
    \begin{itemize}
        \item $\Sigma$ è uno spazio delle fasi
        \item $\mathcal T$ è un insieme del tempo
        \item $\mathcal U \subseteq \R^n$ è l'insieme dei controlli ammessi
        \item $\phi$ è un applicazione detta \textbf{mappa di transizione} del problema:
            \begin{equation*}
                  \begin{array}{cccc}%
                      \phi: &D_\phi &\to &\Sigma \\
                      &\phi^t(\b x_0,\omega) &\mapsto &\b x(t)
                  \end{array}%
            \end{equation*}
            con $D_\phi$ dato da
            \begin{equation*}
                D_\phi = \left\{(t, \b x, \omega) | t \in \mathcal T, \b x \in \Sigma, \omega \in \mathcal U^{[0, t[ \subseteq \mathcal T} \right\}
            \end{equation*}
    \end{itemize}
    definisce un \textbf{problema di controllo} se e solo se valgono le seguenti proprietà:
    \begin{itemize}
        \item \textbf{Identità:}
            $\phi^0(\b x_0, \omega) = \b x_0$
        \item \textbf{Composizione e restrizione:}
            Se $\phi^t(\b x_0, \omega_1) = \b x_1$ e $\phi^s(\b x_1, \omega_2) = \b x_2$
            allora $\phi^{t+s}(\b x_0, \omega) = \b x_2$, con $\omega = \omega_1 \circ \omega_2$.
            È valido anche il contrario, ovvero, se $\phi^{t+s}(\b x_0, \omega) = \b x_2$, posso
            scrivere $\omega = \omega_1 \circ \omega_2$ per cui vale $\phi^t(\b x_0, \omega_1) = \b x_1$ e $\phi^s(\b x_1, \omega_2) = \b x_2$.
        \item \textbf{Non-trivialità:}
            Per ogni stato $\b x \in \Sigma$ esiste sempre un tempo $t$ e un
            controllo $\omega$ per cui $(t, \b x, \omega) \in D_\phi$
    \end{itemize}
    \label{def:problema-di-controllo}
\end{definition}
\todo{Questa definizione l'ho presa dal sontag e ho cercato di alleggerire appena la notazione, rendendola simile
alla definizione di sistema dinamico. Il concetto alla base è lo stesso, spero che vada bene lo stesso.}

Per descrivere un problema di controllo, posso usare un'equazione del moto con la stessa forma
della~\eqref{eq:sistema-non-lineare}:
\begin{equation*}
    \dot {\b x} = \b a(\b x, \b u),\ \text{con } \b x = \b x(t)\ \text e\ \b u = \b u(t).
\end{equation*}
Come ho mostrato nel paragrafo~\ref{subsec:linearizzazione}, è possibile linearizzare
una generica funzione $\b a(\b x, \b u)$ attorno a un punto fisso per ricondurmi alla
forma~\eqref{eq:sistema-linearizzato}.
In questo caso, $\b x$ è un punto dello spazio delle fasi e $B\b u = \omega$ è la funzione
di controllo.

Perché un problema di controllo sia ben posto, è necessario fissare un obiettivo.
In generale, l'obiettivo che ci si pone è trovare una funzione di controllo $\omega$
(o $\b u$ per i sistemi lineari) che alteri l'evoluzione dello stato del sistema $\b x$ a piacimento.
Nel paragrafo~\ref{subsec:condizioni-controllabilità} troverò una condizione sufficiente
alla realizzazione di questo scopo.
Prima di proseguire, voglio sottolineare una precisazione: nella definizione~\ref{def:problema-di-controllo}
ho preso $\omega$ funzione del tempo.
In realtà, in questo testo assumerò che conoscere lo stato del sistema $\b x$
a un certo istante di tempo $t$ sia sufficiente a determinare la funzione di controllo
per quell'istante. $\omega$ deve quindi essere vista come funzione dello stato
del sistema:
\begin{equation*}
    \omega = \omega(\b x(t)).
\end{equation*}

\subsection{Condizioni per la controllabilità}
\label{subsec:condizioni-controllabilità}
Intuitivamente, la nozione di controllabilità riguarda la possibilità di portare il sistema
in uno stato arbitrario, partendo da una qualsiasi condizione iniziale.
Ne enuncio la definizione.
\begin{definition}
    Un problema di controllo descritto dall'equazione del moto~\ref{eq:sistema-non-lineare}
    è detto \textbf{controllabile} se per qualsiasi $t \in \mathcal T$ e per qualsiasi $\b x_0, \b x_1 \in \Sigma$
    esiste una funzione di controllo $\omega \in  \mathcal U^{[0, t[ \in \mathcal T}$ tale che
    \begin{equation*}
        \phi^t(\b x_0, \omega) = \b x_1.
    \end{equation*}
    \label{def:controllabilità}
\end{definition}

Per un sistema lineare nella forma~\eqref{eq:sistema-linearizzato},
la controllabilità dipende solamente dalla \emph{matrice di controllabilità} del sistema, definita come segue.
\begin{definition}
    Dato un sistema nella forma~\eqref{eq:sistema-linearizzato}, con $A \in \M_{n\times n}(\R), B \in \M_{n\times m}(\R)$,
    la matrice
    \begin{equation*}
        \mathcal C = \left(B, AB, A^2B, \ldots, A^{n-1}B \right) \in \M_{n\times (mn)}(\R)
    \end{equation*}
    è detta \textbf{matrice di controllabilità} del sistema.
    \label{def:matrice-controllabilità}
\end{definition}
Nella definizione~\ref{def:matrice-controllabilità}, la matrice $\mathcal C$ è
da intendere come accostamento delle matrici $n \times m$ date dal prodotto delle
potenze di $A$ per $B$.
Dimostro ora una proposizione che fornisce una condizione necessaria e sufficiente per
la controllabilità di un sistema lineare a tempo continuo.

\begin{prop}
    Un sistema nella forma~\eqref{eq:sistema-linearizzato} è controllabile
    se e solo se la sua matrice di controllabilità ha rango massimo.
    \label{prop:condizione-controllabilità}
\end{prop}
\emph{Dimostrazione $\left( \Longleftarrow \right)$.}
Vale l'ipotesi $\rank(\mathcal C) = n$.
La dimostrazione si basa sulla costruzione di una strategia di controllo $\b u(t)$
che soddisfi la definizione~\ref{def:controllabilità} di controllabilità.
Per chiarezza, divido la dimostrazione in più passaggi.
\begin{steps}
    \item Definisco il \emph{Gramiano di controllabilità} del sistema
        \begin{equation}
            W_{\mathcal C} = \int_0^t e^{-As} BB^{\T} e^{-A^\T s}\ ds \in \M_{n\times n} (\R),
            \label{eq:controllability-gramian}
        \end{equation}
    dove ho usato il simbolo $\T$ a esponente per indicare la matrice trasposta.

    \item Dimostro che l'ipotesi implica l'invertibilità di $W_{\mathcal C}$, ovvero,
    $\Ker \Wc = \{\b 0\}$.
    Sia $\b a \in \Ker \Wc$.
    Posso scrivere
    \begin{align*}
        \b 0 &= \b a^\T \Wc \b a \\
        &= \int_0^t \b a^\T \Wc \b a \ ds \\
        &= \int_0^t \b a^{\T}\ e^{-As} B B^\T e^{-A^\T s} \ \b a \ ds \\
        &= \int_0^t \left\| B^\T e^{-A^\T s} \b a \right\|^2\ ds
    \end{align*}
    che implica che la funzione integranda debba essere identicamente nulla nell'intervallo di
    integrazione:
    \begin{equation}
        \b 0 = B^\T e^{-A^\T s} \b a  \text{ per } 0 \leq s \leq t.
        \label{eq:bteata}
    \end{equation}
    Ora considero la~\eqref{eq:bteata} e le sue prime $n-1$ derivate rispetto
    a $s$, che posso esprimere con
    \begin{align*}
        \b 0 &= B^{\T} (A^l)^{\T} e^{-A^\T s} \b a\\
        &= (A^l B)^{\T}e^{-A^\T s} \b a , \text{ con } l = {0, 1, \ldots, n}. \numberthis \label{eq:bteata-derivate}
    \end{align*}
    Le~\eqref{eq:bteata-derivate} devono essere vere per ogni $s$ nell'intervallo $0 \leq s \leq t$.
    Le valuto a $s = 0$ e ottengo
    \begin{equation}
        \b 0 = (A^l B)^{\T} \b a , \text{ con } l = {0, 1, \ldots, n}.
        \label{eq:albta}
    \end{equation}
    La~\eqref{eq:albta} può essere riscritta tramite la matrice di controllabilità del sistema:
    \begin{equation*}
        \b 0 = \mathcal C^{\T} \b a
    \end{equation*}
    e questo implica che $\b a \in \Ker{\mathcal C}$.
    Ma per ipotesi $\Ker{\mathcal C} = \{\b 0\}$, quindi $\Ker \Wc \ni \b a = \b 0$.
    Dall'arbitrarietà nella scelta di $\b a$ ne consegue che $\Ker \Wc$ è formato solo
    dal vettore nullo e quindi $\Wc$ è invertibile.

    \item Dimostro che l'invertibilità di $W_{\mathcal C}$ implica la controllabilità del sistema.
    Scelgo come strategia di controllo
    \begin{equation*}
        \b u(s) = B^\T e^{-A^\T s} \Wc^{-1} e^{-At} (\b x_1 - \b x_0) \in \mathcal U^{[0, t[}
    \end{equation*}
    con $\b x_0 = \b x(0)$ e $\b x_1$ è lo stato del sistema che voglio raggiungere al tempo $t$.
    Applico la~\ref{eq:soluzione-lineare-non-omogeneo}:
    \begin{align*}
        \b x(t) &= \b x_0+ \int_0^t e^{A(t-s)} B \b u(s)\ ds \\
                &= \b x_0 + (\b x_1 - \b x_0) \int_0^t e^{A(t-s)} B B^\T e^{-A^\T s} \Wc^{-1} e^{-At}\ ds \\
                &= \b x_0 + (\b x_1 - \b x_0) e^{At} e^{-At} \left(\int_0^t e^{-As} B B^\T e^{-A^\T s}\ ds \right)\Wc^{-1}. \numberthis \label{eq:x0eatint}
    \end{align*}
    Nella~\eqref{eq:x0eatint} compare la definizione di Gramiano~\eqref{eq:controllability-gramian}
    che si semplifica con il termine~$\Wc^{-1}$.
    Anche i rimanenti due termini esponenziali si semplificano e si ottiene la soluzione
    \begin{equation*}
        \b x(t) = \b x_1.
    \end{equation*}
\end{steps}
Di conseguenza, fissato uno stato arbitrario $x_1$, è possibile raggiungerlo in un tempo finito $t$
partendo da qualsiasi condizione iniziale $\b x_0$ e quindi il sistema è controllabile.
\hfill\qedsymbol

\emph{Dimostrazione $\left( \Longrightarrow \right)$.}
Inizio dimostrando il seguente lemma.
\begin{lemma}
    ciao
\end{lemma}

%todo
\hfill\qedsymbol

%todo spiegare che questo cale anche per sistemi a tempo discreto.

L'esempio~\ref{ex:controllabilità} mostra due casi estremamente semplificati
di sistemi rispettivamente controllabili e non controllaibli.
\begin{example}
    %todo
    \label{ex:controllabilità}
\end{example}

\subsection{Pole placement}
\section{Controllo ottimale: LQR} \todo {glossary LQR}

\subsection{Princièio di Maxima}
(qui ci metto la definizione generale)

\subsection{Costo quadratico}

\subsection{Equazione di Riccati}