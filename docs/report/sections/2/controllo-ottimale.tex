\section{Controllo ottimale: il Regolatore Lineare Quadratico} \todo {glossary LQR}
\label{sec:controllo-ottimale}
\subsection{Problema di controllo ottimale}
\begin{definition}
    Un problema di controllo per cui è definita una funzione costo
    \begin{equation*}
        J(t) = \phi(\b x) + \int_{\mathcal T} l(\b x, \omega) \ dt
    \end{equation*}
    dove $\phi : \Sigma \to \R$ è una funzione che misura il costo dello stato finale
    e $l : \Sigma \times \mathcal U \to \R$ è una funzione che rappresenta il costo
    della deviazione dallo stato finale per ogni istante di tempo
    è detto \textbf{problema di controllo ottimale}.
\end{definition}
Risolvere un problema di controllo ottimale equivale a ricavare una strategia
di controllo $\omega$ che ne minimizzi la funzione costo.

\subsection{Regolatore lineare quadratico}
Considero un sistema lineare nella forma~\eqref{eq:sistema-lineare-non-omogeneo} controllabile.
Posso formulare un problema di controllo ottimale definendo una funzione costo.
\begin{definition}
    Sia dato un problema di controllo lineare nella forma~\eqref{eq:sistema-lineare-non-omogeneo}.
    Siano $Q \in \M_{n\times n}(\R)$ una matrice semidefinita positiva, $R \in \M_{m\times m}(\R)$ una matrice definita positiva
    e sia definita la funzione costo
    \begin{equation}
        J(t) = \int_0^t
            \left( \b x^{\T} Q \b x + \b u^\T R \b u \right)
        \ ds.
        \label{eq:lqr-cost}
    \end{equation}
    con $\b x = \b x(s)$ e $\b u = \b u(s)$.
    La strategia di controllo $\b u = K\b x$ con $K \in \M_{n\times m}(\R)$ scelta in modo da
    minimizzare la funzione costo~\eqref{eq:lqr-cost} per $t \to +\infty$ è
    detta \textbf{regolatore lineare quadratico (\textsc{LQR}}).
    \label{def:lqr}
\end{definition}
Le matrici $Q$ ed $R$ nella \autoref{def:lqr} rappresentano rispettivamenge
il costo della deviazione dello stato del sistema da $\b 0$ e il costo dell'attuazione del
controllo.
Nel paragrafo~\ref{subsec:derivazione-riccati} mostrerò che la soluzione per $K$ è
\begin{equation}
    K = R^{-1}BP
    \label{eq:riccati-K}
\end{equation}
dove $P$ è la soluzione all'equazione di Riccati
\begin{equation}
        A^{\T} P + P A - P B R^{-1} B^{\T} P + Q = \b 0.
    \label{eq:riccati}
\end{equation}
La soluzione della~\eqref{eq:riccati} si può trovare numericamente in tempo polinomiale
in $n$~\cite{riccatiO3}.
Nel corso di questo testo, userò la funzione \verb|RiccatiSolve| del software
\emph{Mathematica}~\cite{Mathematica} per risolvere la~\eqref{eq:riccati}.

\subsection{Derivazione dell'equazione di Riccati}

\label{subsec:derivazione-riccati}
Devo minimizzare la~\eqref{eq:lqr-cost} lungo le traiettorie del sistema.
Uso il metodo dei moltiplicatori di Lagrange e impongo che la variazione totale sia nulla.
\begin{equation*}
    \delta J(t) = \delta \int_0^t
    \left[ \b x^{\T} Q \b x + \b u^\T R \b u  - \b \lambda(\dot{ \b x} - A \b x - B\b u)
    \right]\ ds = \b 0.
    \label{eq:variation-is-zero}
\end{equation*}
La funzione integranda nella~\eqref{eq:variation-is-zero} deve
soddisfare le equazioni di Eulero-Lagrange.
Per chiarezza la indico con $f$ e la riscrivo in notazione indiciale,
in cui sottointendo la somma sugli indici ripetuti.
\begin{equation*}
    f = x_i Q_{ij} x_j + u_l R_{lk} u_k - \lambda_i(\dot x_i - A_{ij} x_j - B_{ik} u_k).
\end{equation*}
Le equazioni di Eulero-Lagrange sono
\begin{equation*}
    \left\{
    \begin{aligned}
        \totald t \partiald{\dot x_a} f - \partiald{x_a} f &= 0, \text{ con } a = 0, \ldots, n, \\
        \totald t \partiald{\dot u_b} f - \partiald{u_b} f &= 0, \text{ con } b = 0, \ldots, m, \\
        \totald t \partiald{\dot \lambda_c} f - \partiald{\lambda_c} f &= 0, \text{ con } c = 0, \ldots, n
    \end{aligned}
    \right.
\end{equation*}
e calcolando le derivate ottengo

\begin{subequations}
    \begin{empheq}[left=\empheqlbrace]{align}
        \dot \lambda_a + 2Q_{aj} x_j + \lambda_i A_{ia} &= 0, \text{ con } a = 0, \ldots, n, \label{eq:sistema-variazionale-1} \\
        2R_{bj}u_j - \lambda_i B_{ib} &= 0, \text{ con } b = 0, \ldots, m, \label{eq:sistema-variazionale-2}\\
        \dot x_c - A_{cj}x_j - B_{ci}u_i &= 0, \text{ con } c = 0, \ldots, n. \label{eq:sistema-variazionale-3}
    \end{empheq}
    \label{eq:sistema-variazionale}
\end{subequations}
Per risolvere il sistema~\eqref{eq:sistema-variazionale} faccio un ansatz
\begin{equation}
    \lambda_a = P_{aj}x_j,
    \label{eq:ansatz-lambda}
\end{equation}
inserisco la~\eqref{eq:ansatz-lambda} nella~\eqref{eq:sistema-variazionale-1} e~\eqref{eq:sistema-variazionale-2} e ottengo
\begin{subequations}
    \begin{align}
        \dot P_{ai}x_i + P_{ai} \dot x_i + 2Q_{ai} x_i + \lambda_i A_{ia} &= 0, \label{eq:sistema-variazionale-4} \\
        2R_{bk}u_k - B_{ib} P_{ij}x_j &= 0 \label{eq:sistema-variazionale-5}
    \end{align}
\end{subequations}
 La~\eqref{eq:sistema-variazionale-3} coincide con l'equazione del moto.
La uso per sostituire $\dot x_j$ nella~\eqref{eq:sistema-variazionale-4}
\begin{equation}
    \dot P_{ai}x_i + P_{ai} (A_{ij}x_j + B_{ik}u_k) + 2Q_{ai} x_i + A_{ia}P_{ij}x_j  = 0.
    \label{eq:sistema-variazionale-6}
\end{equation}
Ricavo $u_j$ dalla~\eqref{eq:sistema-variazionale-5}
\begin{align*}
    2R^{-1}_{lb}R_{bk}u_k &= R^{-1}_{lb} B_{ib} P_{ij}x_j  \\
    \delta_{lk} u_k &= \frac 1 2 R^{-1}_{lb} B_{ib} P_{ij}x_j \numberthis \label{eq:riccati-u}
\end{align*}
e lo sostituisco nella~\eqref{eq:sistema-variazionale-6}.
\begin{equation}
    \dot P_{ai}x_i + P_{ai} (A_{ij}x_j + \frac 1 2  B_{ik} R^{-1}_{kb} B_{mb} P_{mj}x_j) + 2Q_{ai} x_i + A_{ia} P_{ij}x_j = 0.
    \label{eq:sistema-variazionale-7}
\end{equation}
Nella~\eqref{eq:sistema-variazionale-7} l'unico indice libero è $a$.
La riscrivo in notazione matriciale
\begin{equation*}
    \dot P \b x + PA \b x + A^\T P\b x +  \frac 1 2 P B R^{-1} B^\T \b x + 2 Q \b x = 0.
\end{equation*}
Raccogliendo $\b x$ e imponendo $\dot P \to 0$ per $t \to +\infty$ si riottiene
la~\eqref{eq:riccati}. I fattori $2$ e $\frac 1 2$ possono essere rimossi
definendo $P = 2P'$ e dividendo tutto per $2$.
Per ottenere la~\eqref{eq:riccati-K} basta riscrivere
la~\eqref{eq:riccati-u} in forma matriciale.

\hfill\qedsymbol
