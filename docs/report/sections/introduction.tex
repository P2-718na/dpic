\title{Introduzione}
\maketitle
\label{sec:intro}


\section{Teoria del controllo}

\paragraph{È facile come andare in bicicletta.}
Questo proverbio viene spesso
usato quando si vuole sottolineare che un certo compito è \emph{molto facile}.
Infatti, sin da bambini, si può imparare ad usare la bicicletta con facilità e,
al giorno d'oggi, la maggior parte degli italiani\cite{https://www.ipsos.com/sites/default/files/ct/news/documents/2022-05/Ipsos\%20-\%20Cycling\%20Across\%20the\%20World-2022.pdf}
sa andare in bici.
Tuttavia, indagando più in profondità, si scopre che la matematica che ci
permette di svolgere un compito così semplice non è affatto banale\cite{a}.
Infatti, a differenza di ciò che spesso si pensa, non è sufficiente affidarsi
all'effetto giroscopico delle ruote per rimanere stabili, ma bisogna attivamente
\emph{controllare} la bici, agendo sul manubrio e spostando il proprio peso corporeo.
La moderna \empgh{Teoria del Controllo} è la scienza che ci permette di studiare
analiticamente fenomeni di questo tipo, riconciliando la dualità che spesso
si osserva nella forma di \emph{compito-facile $\iff$ matematica-non-banale}.

\paragraph{La necessità di avere Controllo} è ampiamente diffusa in Natura\cite[b].
Basti pensare che tutti gli organismi viventi necessitano di meccanismi per controllare
il proprio metabolismo.
Meccanismi diversi permettono al singolo organismo di muoversi e agire secondo le sue decisioni.
Altri meccanismi regolano l'interazione tra organismi diversi della stessa specie.
Tutti questi fenomeni di natura diversa possono essere studiati tramite la stessa
formulazione matematica.

Per fissare le idee, consideriamo un sistema fisico descritto dall'equazione
\begin{equation*}
    \b A( \b y) = \b f( \b u)
\end{equation}
. $\b y \in Y$ è lo \emph{stato} e appartiene a uno spazio vettoriale
$Y$. $u \in \mathcal U_a$ è il \emph{controllo} e appartiene all'insieme dei
\emph{controlli ammissibili} $\mathcal U_a$.
\todo{bla bla ricopia paper storia swag fino a subito prima parte di feedback}


\subsection{Alcuni concetti fondamentali}

\paragraph{Open loop}

\paragraph{Closed loop}

\subsection{Controllabilità dei sistemi}

\section{Il "doppio pendolo invertito su rotaia"}

