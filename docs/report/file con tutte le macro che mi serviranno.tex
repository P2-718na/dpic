

\todo[inline]{Rewrite this!}
% Standalone with \input:
\begin{figure}[htbp]
    \centering
    %\input{figures/ball}
    \caption[One ball]{One ball.}
\end{figure}

% Standalone with \includegraphics:
\begin{figure}[thbp]
    \centering
    %\includegraphics{balls}
    \caption[Two balls]{Two balls.}
\end{figure}

% Todonotes:
\begin{figure}[hbp]
    \centering
    \missingfigure{Three balls.}
    \caption[Three balls]{Three balls.}
\end{figure}


% Booktabs:
\begin{table}[htbp]
    \centering
    \begin{tabular}{@{}ll@{}}
        \toprule
        \textbf{Correct}               & \textbf{Incorrect}      \\
        \midrule
        \( \varphi \colon X \to Y \)   & \( \varphi : X \to Y \) \\[0.5ex]
        \( \varphi(x) \coloneqq x^2 \) & \( \varphi(x) := x^2 \) \\
        \bottomrule
    \end{tabular}
    \caption[Colons]{Proper colon usage.}
\end{table}

\begin{table}[htbp]
    \centering
    \begin{tabular}{@{}ll@{}}
        \toprule
        \textbf{Correct}     & \textbf{Incorrect}         \\
        \midrule
        \( A \implies B \)   & \( A \Rightarrow B \)      \\
        \( A \impliedby B \) & \( A \Leftarrow B \)       \\
        \( A \iff B \)       & \( A \Leftrightarrow B \)  \\
        \bottomrule
    \end{tabular}
    \caption[Arrows]{Proper arrow usage.}
\end{table}

% Tablefootnote and multirow:
\begin{table}[htbp]
    \centering
    \begin{tabular}{@{}ll@{}}
        \toprule
        \textbf{Correct}
        &
        \textbf{Incorrect}
        \\
        \midrule
        \( -1 \)
        &
        -1
        \\[0.3ex]
        1--10
        &
        1-10
        \\[0.3ex]
        Birch--Swinnerton-Dyer\tablefootnote{It is now easy to tell that Birch and Swinnerton-Dyer are two people.} conjecture
        &
        Birch-Swinnerton-Dyer conjecture
        \\[0.3ex]
        The ball \dash which is blue \dash is round.
        &
        \multirow{ 2}{*}{The ball - which is blue - is round.}
        \\[0.3ex]
        The ball---which is blue---is round.
        &
        \\
        \bottomrule
    \end{tabular}
    \caption[Dashes]{Proper dash usage.}
\end{table}

\begin{table}[hbtp]
    \centering
    \begin{tabular}{@{}*{2}{p{0.5\textwidth}}@{}}
        \toprule
        \textbf{Correct} &  \textbf{Incorrect}
        \\
        \midrule
        \enquote{This is an \enquote{inner quote} inside an outer quote}
        &
        "This is an 'inner quote' inside an outer quote"
        \\
        \bottomrule
    \end{tabular}
    \caption[Quotation marks]
    {Proper quotation mark usage.
    The \texttt{\textbackslash enquote} command chooses the correct
    quotation marks for the specified language.}
\end{table}

\section{Summary of Papers}

\begin{description}
    \item[\cref{pap:first}]
    focuses on the aspects of being the first paper of a thesis,
    following \cref{sec:intro}.

    \item[\cref{pap:A4}]
    demonstrates how illegible the font size becomes when an A4 paper article is shrunk in order to fit into the thesis.

    \item[\cref{pap:third}]
    shows a new and exciting result about the final paper in an article based doctoral thesis.
\end{description}

The \Gls{latex} typesetting markup language is specially suitable
for documents that include \gls{maths}.


% per aggiungere indice, subito dopo \maketitle in \chapter
\startcontents[chapters]
\printcontents[chapters]{}{1}{\section*{\contentsname}}
%% Format bibliography like a section, not a chapter:
%\printbibliography[heading = subbibliography]
\stopcontents[chapters] %da mettere alla fine dlel'indice

% subito prima di \maketitle, per inserire testo
\metadata
{
    Ciao
}

\foreach \env in {theorem, corollary, lemma, proposition, observation,
conjecture, definition, example, notation, remark}